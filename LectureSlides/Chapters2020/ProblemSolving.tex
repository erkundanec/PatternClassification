\section{Assignment}
\subsection{}
\begin{frame}{}
\begin{variableblock}{\centering \Large \textbf{\vspace{4pt}\newline Assignment\vspace{4pt}}}{bg=slidecolor,fg=white}{bg=slidecolor,fg=white}
\end{variableblock}
\end{frame}

\begin{frame}{Assignment}
\textit{\color{slidecolor}Question 01}:
\begin{itemize}
\item[i.] What is the order of the shape number for figure shown?
\item[ii.] Obtain the shape number.
\begin{figure}
\includegraphics[scale=0.8]{Question01}
\end{figure}
\end{itemize}
\textit{\color{slidecolor}Question 02}: Draw the Medial axis of the following
\begin{itemize}
\item[i.] Circle
\item[ii.] Square
\item[iii.] Rectangle
\item[iv.] Equilateral Triangle
\end{itemize}
\end{frame}

\begin{frame}{Assignment}
\textit{\color{slidecolor}Question 03}:
\begin{itemize}
\item[i.] Compute gray level co-occurance matrix $G_{1,45^o}$ for given image.
\begin{figure}
\includegraphics[scale=0.25]{GLCM01}
\end{figure}
\item[ii.] Compute probability density function of GLCM matrix.
\item[iii.] Compute Maximum probability, Correlation, Contrast, Uniformity (Energy), Homogeneity, and Entropy features.
\end{itemize}
\end{frame}

\begin{frame}{Assignment}
\textit{\color{slidecolor}Question 04}: Consider a binary image of size pixels, with a vertical black band extending from columns 1 to 99 and a vertical white band extending from columns 100 to 200
\begin{itemize}
\item[i.] Obtain the co-occurrence matrix of this image using the position operator
``one pixel to the right.''
\item[ii.]Normalize this matrix so that its elements become probability estimates.
\item[iii.] Maximum probability, Correlation, Contrast, Uniformity (Energy), Homogeneity, and Entropy features.
\end{itemize}
\end{frame}

\begin{frame}{Assignment}
\textit{\color{slidecolor}Question 05}: Draw the signature of the following shapes:
\begin{itemize}
\item[i.] Equilateral Triangle
\item[ii.] Rectangle
\item[iii.] Ellipse
\end{itemize}
\textit{\color{slidecolor}Question 06}:
\begin{itemize}
\item[i.] Show that the first difference of a chain code normalizes it to rotation
\item[ii.] Compute the first difference of the code 0101030303323232212111
\end{itemize}
\end{frame}



\begin{frame}{Assignment}
\textit{\color{slidecolor}Question 07:} What is the probability that a person has a cold given that he or she has fever if the prior probability of a person having cold is 0.01, and the probability of having a fever, given that the person has a cold, is 0.4, and the probability of fever in the general population (including people with and without cold) is 0.02.
\end{frame}

\begin{frame}{Assignment}
\textit{\color{slidecolor}Question 08:} {Detecting the \textit{HIV} virus using the \textit{ELISA (enzyme-linked immunosorbent assay)} test.}\\
What is the probability that a patient has the HIV virus if the test result is positive?
Assuming the following probabilities for patients in the clinic:\\
\hspace{2cm}$P(H)=0.15$\\
\hspace{2cm}$P(\bar{H})=0.85$\\
\hspace{2cm}$P(Pos|{H})=0.95$\\
\hspace{2cm}$P(Pos|\bar{H})=0.02$\\
where $H$ be the event that a patient has the HIV virus, $\bar{H}$ be the event that a patient does not have the virus, $Pos$ is the event that the patient tests positive for the virus, and $Neg$ be the event that the test result is negative.
\end{frame}

\begin{frame}{Assignment}
\textit{\color{slidecolor}Question 09:} {\color{slidecolor}Classification of college applicants by ACT scores}\\
According to records at Eastsoutheastern State University (ESU), the probability that a student will graduate withing five years after begining classes as a freshman is 0.8. The American College Testing Service test scores (ACT scores) of those who graduate with five years are normally distributed with a mean of 26 and a standard deviation of 2. The scores of those who do not graduated within five years is also normally distributed but with a mean of 22 and a standard deviation of 3. Since one of these two mutually exclusive events must occur for each admitted student, the students admitted five or more year ago can be divided into two classes: $G$ and $\bar{G}$. If the student with an ACT score of 22 is admitted, what is the probability that he or she will graduate within five years?
\end{frame}

\begin{frame}{Assignment}
\textit{\color{slidecolor}Question 10:} {\color{slidecolor}Computing optimal one-dimensional decision boundaries.}\\
To compute the decision boundaries for previous example. Data given $\mu_{G}=26$, $\sigma_G=2$, $\mu_{\bar{G}}=22$, $\sigma_{\bar{G}}=3$, $P(G)=0.8$, and $P(\bar{G})=0.2$.
\end{frame}

\begin{frame}{Assignment}
\textit{\color{slidecolor}Question 11:} {\color{slidecolor}Classification of college applicants by ACT scores and class rank}\\
Suppose that in addition to classifying college application by their ACT score $x$, Eastsoutheastern University also uses $y$, the percentile rank in an applicant's high school graduating class. Assume that the conditional densities of both $x$ and $y$ given the class are bivariate normal. (Since $y$ is a percentile, it is uniformly distributed for the graduates of any particular graduating class, but this does not prevent $y$ from being normally distributed for the applicants to a particular university.) The parameters for class $G$ are $P(G)=0.8$, $\mu_x=26$, $\sigma_x=2$, $\mu_y=85$, $\sigma_y=5$ and $\rho_{xy} =0.6$. The parameter for class $\bar{G}$ are $P(\bar{G})=0.2$, $\mu_x=22$, $\sigma_x=3$, $\mu_y=70$, $\sigma_y=8$ and $\rho_{xy} =0.5$. What is the probability that an applicant with ACT score 22 and class rank 70 will graduate within five years?
\end{frame}

\begin{frame}{Assignment}
\textit{\color{slidecolor}Question 12:} {\color{slidecolor}Classification in 3-D space using three features}\\
Given ${\mu _G} = \left( {\begin{array}{*{20}{c}}
{26}\\
{85}\\
8
\end{array}} \right)$, ${\mu _{\bar{G}}} = \left( {\begin{array}{*{20}{c}}
{22}\\
{70}\\
6
\end{array}} \right)$, ${\Sigma _G} = \left( {\begin{array}{*{20}{c}}
4&6&3\\
6&{25}&5\\
3&5&4
\end{array}} \right)$, and ${\Sigma _G} = \left( {\begin{array}{*{20}{c}}
9&12&8\\
12&{64}&16\\
8&16&4
\end{array}} \right)$. If an incoming student has ACT score 22, class rank 70, and a 6 on the essay submitted with the application, the feature vector is ${\rm x}=[22~~70~~6]^T$
\end{frame}

\begin{frame}{Assignment}
\textit{\color{slidecolor}Question 13:} {\color{slidecolor}Determining the optimal decision boundary between two simple bivariate normal classes when the features are independent within each class}\\
Assume the the classes $A$ and $B$ are both bivariate normal with prior probabilities $P(A)=P(B)=0.5$. The parameters for the conditional density of class $A$ are
\begin{equation}
\mu_x=0,~\mu_y=0~,\sigma_x=1,~\sigma_y=1,~\rho_{xy}=0,\nonumber
\end{equation}
and the parameter for class $B$ are
\begin{equation}
\mu_x=2,~\mu_y=0~,\sigma_x=1,~\sigma_y=2,~\rho_{xy}=0,\nonumber
\end{equation}
\end{frame}

\begin{frame}{Assignment}
\textit{\color{slidecolor}Question 14:} {\color{slidecolor}Determining the optimal decision boundary between two bivariate normal classes.}\\
Assume the the classes $A$ and $B$ are both bivariate normal with prior probabilities $P(A)=0.8$, and $P(B)=0.2$. The parameters for the conditional density of class $A$ are
\begin{equation}
\mu_x=26,~\mu_y=2~,\sigma_x=85,~\sigma_y=5,~\rho_{xy}=0.6,\nonumber
\end{equation}
and the parameter for class $B$ are
\begin{equation}
\mu_x=22,~\mu_y=3~,\sigma_x=70,~\sigma_y=8,~\rho_{xy}=0.5,\nonumber
\end{equation}
\end{frame}

\begin{frame}{Assignment}
\textit{\color{slidecolor}Question 15:} {\color{slidecolor}Determining the optimal decision boundary when the two classes have the same variances and correlations}\\
For this example, assume that $P(G)=0.8$, $\mu_x=26$, and $\mu_y=85$ for class $G$, and that $P(\bar{G})=0.2$, $\mu_x=22$, $\mu_y=70$ for class $\bar{G}$ as before; but assume $\sigma_x=2$, $\sigma_y=5$, and $\rho_{xy}=0.6$ for both classes $G$ and $\bar{G}$.
\end{frame}

\begin{frame}{Assignment}
\textit{\color{slidecolor}Question 16:} {\color{slidecolor}Finding the decision boundary between two multivariate normal classes with equal covariance matrices}\\
Given ${\mu _G} = \left( {\begin{array}{*{20}{c}}
{26}\\
{85}\\
8
\end{array}} \right)$, ${\mu _{\bar{G}}} = \left( {\begin{array}{*{20}{c}}
{22}\\
{70}\\
6
\end{array}} \right)$, ${\Sigma _G} ={\Sigma _{\bar{G}}}= \left( {\begin{array}{*{20}{c}}
4&6&3\\
6&{25}&5\\
3&5&4
\end{array}} \right)$. \\The prior probabilities are $P(G)=0.8$ and $P(\bar{G})=0.2$
\end{frame}

\begin{frame}{Assignment}
\textit{\color{slidecolor}Question 17:} {\color{slidecolor}Choosing the class that minimizes the risk}\\
Suppose that in a two-class problem, the reward for correctly classifying a sample from class $A$ is $\$$3 and for correctly classifying a sample from class $B$ is $\$$4 ($L(\hat{A}|A)=-3$ and $L(\hat{B}|B)=-4$). The penalty for misclassifying an $A$ is $\$$8 ($L(\hat{B}|A)=8$) and for misclassifying a $B$ is $\$$20 ($L(\hat{A}|B)=-20$). Also assume that for some particular sample with feature value $x$, $P(A/x)=0.6$. There are only two classes, so $P(B/x)=0.4$. Chose the class that minimizes the risk.
\end{frame}

\begin{frame}{Assignment}
\textit{\color{slidecolor}Question 18: Compute the optimal decision.}\\
$\omega=\{\omega_1,\omega_2\}$\\
$P({\rm x}/\omega_1)\Rightarrow N(2,0.5)$ (Normal distribution)\\
$P({\rm x}/\omega_2)\Rightarrow N(1.5,0.2)$\\
$P(\omega_1)=2/3$ and $P(\omega_2)=1/3$ \\
$\lambda = \left[ {\begin{array}{*{20}{c}}
1&2\\
3&4
\end{array}} \right]$
\end{frame}

\begin{frame}{Assignment}
\textit{\color{slidecolor}Question 19: Compute Bayesian decision for three-dimensional binary features}\\
Suppose two categories consist of independent binary features in three dimensions
with known feature probabilities. Let us construct the Bayesian decision boundary if
$P(\omega_1 ) = P(\omega_2 )=0.5$ and the individual components obey:
\begin{equation}
\left\{ {\begin{array}{*{20}{c}}
{{p_i} = 0.8}\\
{{q_i} = 0.5}
\end{array}} \right.~~~~~~i = 1,2,3\nonumber
\end{equation}
\end{frame}

\begin{frame}{Assignment}
\textit{\color{slidecolor}Question 20:Compute Bayesian decision for three-dimensional binary features}\\
Suppose two categories consist of independent binary features in three dimensions
with known feature probabilities. Let us construct the Bayesian decision boundary if
$P(\omega_1 ) = P(\omega_2 )=0.5$ and the individual components obey:
\begin{equation}
\left\{ {\begin{array}{*{20}{c}}
{{p_1} = {p_2}=0.8},~p_3=0.5\\
{{q_1} = {q_2} = {q_3} = 0.5}
\end{array}} \right.\nonumber
\end{equation}
\end{frame}

